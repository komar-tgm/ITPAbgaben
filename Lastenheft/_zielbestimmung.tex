%!TEX root=../protocol.tex	% Optional

\section{Zielbestimmung}
Für den Anwender soll es möglich sein, eine Buchung vor Ort durchführen zu können. Dabei entscheiden die Ergebnisse der Kosten-/Umsetzbarkeitsanalyse, ob der Anwender nur im Standort (z.B. Stand-Terminal) oder aber auch außerhalb des Standortes (z.B. Wandmontage) buchen kann. Dem Anwender soll eine benutzerfreundliche Buchung ermöglicht werden. Der bereits für Desktop- und Mobilgeräte umgesetzte Buchungsvorgang soll auch am neuen System möglich sein (Touch vs Keyboardterminal etc.). 

\section{Produkteinsatz}
Das Terminal wird ausschließlich zum Buchen eines Abteils genutzt. Es wird vor jedem Standort aufgestellt, sodass direkt vor Ort schnell eine Mietung abgeschlossen werden kann.
\\
Neukunden, Bestandskunden und auch Sales-Manager stellen Anwender des Storebox-Terminals dar. Die Zielgruppe ist mit Anwender von 18 bis 80 Jahren breit gefächert. Besonders älteren Menschen soll eine reibungslose Nutzererfahrung geboten werden.
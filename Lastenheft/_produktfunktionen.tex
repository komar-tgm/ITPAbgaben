%!TEX root=../protocol.tex	% Optional

\section{Produktfunktionen}
\subsection{Benutzer registrieren, an und abmelden}
\subsubsection{Benutzer Registrieren (/LF010/)}
Ein Kunde kann sich auf dem Terminal registrieren. Dafür werden folgende Informationen von ihm benötigt:
\begin{itemize}
    \item E-Mail
    \item Telefonnummer
    \item Passwort
    \item Kundentyp
    \item Anrede
    \item Vorname
    \item Nachname
    \item Geburtsdatum
    \item Adresse
\end{itemize}
\subsubsection{Benutzer anmelden (/LF020/)}
Wenn der Kunde registriert ist, kann er sich am Terminal anmelden. Dafür werden folgende Anmeldedaten benötigt:
\begin{itemize}
    \item E-Mail oder Benutzername
    \item Passwort
\end{itemize}
Nachdem sich der Kunde angemeldet hat, kann er eine Buchung durchführen.
\subsubsection{Passwort vergessen (/LF030/)}
Falls der Kunde sein Passwort vergessen hat, hat er eine Option ein neues Passwort anzufordern.
Dazu wird die \textbf{E-Mail Adresse} benötigt.
\subsubsection{Benutzer abmelden}
Der Benutzer kann sich aus dem Hauptmenü vom Terminal abmelden.
\subsection{Buchung durchführen}
% Auswahl der Storage Größe
% Auswahl der Mietdauer
% Anzahl der Schwerbelastungsregale und Aufbewahrungsboxen
% Auswahl des Versicherungsschutzes

% Für die Bezahlung wird benötigt:
% Zahlungsart (Bankeinzug,Kreditkarte,Diners Club,Paypal oder Rechung) oder mit einem Gutscheincode

% Bankeinzug: IBAN, BIC, Inhaber, Straße, Hausnummer, PLZ, Ort, Land
% Kreditkarte: Kartennummer, Ablaufdatum, Verifizierungscode
% Diners Club: Kartennummer, Ablaufdatum, Verifizierungscode
% Paypal: -
% Rechung: -
% Gutscheincode: Code
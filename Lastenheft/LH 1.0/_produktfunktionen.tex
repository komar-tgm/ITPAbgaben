%!TEX root=../protocol.tex	% Optional

\section{Produktfunktionen}
\subsection{Benutzer registrieren, an und abmelden}
\subsubsection{Benutzer Registrieren (/LF010/)}
Ein Kunde kann sich auf dem Terminal registrieren. Dafür werden folgende Informationen von ihm benötigt:
\begin{itemize}
    \item E-Mail
    \item Telefonnummer
    \item Passwort
    \item Kundentyp
    \item Anrede
    \item Vorname
    \item Nachname
    \item Geburtsdatum
    \item Adresse
\end{itemize}
\subsubsection{Benutzer anmelden (/LF020/)}
Wenn der Kunde registriert ist, kann er sich am Terminal anmelden. Dafür werden folgende Anmeldedaten benötigt:
\begin{itemize}
    \item E-Mail oder Benutzername
    \item Passwort
\end{itemize}
Nachdem sich der Kunde angemeldet hat, kann er eine Buchung durchführen.
\subsubsection{Passwort vergessen (/LF030/)}
Falls der Kunde sein Passwort vergessen hat, hat er eine Option ein neues Passwort anzufordern.
Dazu wird die \textbf{E-Mail Adresse} benötigt.
\subsubsection{Benutzer abmelden}
Der Benutzer kann sich aus dem Hauptmenü vom Terminal abmelden.
\subsection{Buchung durchführen}
Es soll möglich sein am Terminal direkt vor Ort eine Buchung durchzuführen. Dazu muss der Benutzer einen Account haben und angemeldet sein. Für die Buchung sind folgende Informationen wichtig:
\begin{itemize}
    \item Auswahl der Storage Größe
    \item Auswahl der Mietdauer
    \item Auswahl der Schwerbelastungsregale und Aufbewahrungsboxen
    \item Auswahl des Versicherungsschutzes
\end{itemize}
Für die Bezahlung gibt es verschiedene Möglichkeiten. Je nachdem welche Art der Bezahlung man wählt, müssen dementsprechend auch die Informationen angegeben werden. Die Zahlungsarten sind:
\begin{itemize}
    \item Bankeinzug: IBAN, BIC, Inhaber, Straße, Hausnummer, PLZ, Ort, Land
    \item Kreditkarte: Kartennummer, Ablaufdatum, Verifizierungscode
    \item Diners Clun: Kartennummer, Ablaufdatum, Verifizierungscode
    \item PayPal:
    \item Rechnung:
    \item Gutscheincode: Code eingeben
\end{itemize}
\subsection{Kundensupport kontaktieren}
Es soll möglich sein, Hilfe und Support aus der Storebox Zentrale anzufordern. Diese werden dann direkt mit einem Supportmitarbeiter verbunden (via VoiceCall oder VideoChat).
\subsection{Schwarzes Brett anzeigen}
Es soll möglich sein, auf dem Terminal ein Schwarzes Brett anzeigen zu lassen. Diese zeigt die Kontaktmöglichkeiten an.